\chapter*{РЕФЕРАТ}

\bigskip\par
Выпускная квалификационная работа содержит \pageref*{LastPage}~страницы, 37~рисунков,                                        7~таблиц, 28 использованных источника.

\bigskip\par
СЕНТИМЕНТ-АНАЛИЗ, БИНАРНАЯ КЛАССИФИКАЦИЯ, ГЛУБОКОЕ ОБУЧЕНИЕ, ВЕКТОРИЗАЦИЯ СЛОВ, АНСАМБЛИРОВАНИЕ, ТЕМАТИЧЕСКОЕ МОДЕЛИРОВАНИЕ, КЛАСТЕРИЗАЦИЯ

\bigskip\par
Выпускная квалификационная работа посвящена исследованию сентимент-анализа и тематического моделирования текстов с помощью нейросетевых комплексов, кластеризации и вероятностных подходов. В ходе исследования были разработаны ансамблевая модель бинарной классификации тональности текстов, метод латентного размещение Дирихле (LDA), исследован метод, лежащий в основе аппарата тематического моделирования BertTopic, а также разработан веб-сервис визуализации имплементации этих инструментов.

\bigskip
Теоретическая часть работы содержит обзор современных методов решения задачи сентимент-анализа текстов, векторизации слов и тематического моделирования с математической формализацией и анализом использования выбранного инструментария.


\bigskip
В практической части содержится анализ построенных моделей в зависимости от вариации гиперпараметров, ансамблирование моделей для сентимент-анализа и аргументация выбора конечного инструмента решения проблемы.

\bigskip
Результат разработки нейросетевого комплекса для решения задачи сентимент-анализа был представлен на XVI международной отраслевой научно-технической конференции <<Технологии информационного будущего>>. Аргументация применения ансамблирования к смежной задаче бинарной классификации сложных структур была представлена на XLVIII Международной молодёжной научной конференции <<Гагаринские чтения>> \cite{gagar}
