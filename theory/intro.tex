\chapter{\introname}
Актуальность автоматизации семантического анализа 
не способна подвергаться сомнению. Очевидным следствием является
плотная концентрация научных работ, статей и книг, направленных на решение поставленной проблемы.
Для этого предлагались различные варианты: от построения моделей, основанных на правилах,
требующих много потраченного времени и постоянной модерации в силу нестабильности постоянно
мутирующего языка, до методов машинного обучения и нейронных сетей, последним из которых и посвящена
данная работа. С целью проведения экскурса в выбранной сфере рассмотрим некоторые из существующих работ выбранной тематики.\\
В статье \cite{Fang} сконцентрированно разбираются подходы машинного обучения:
наивный байесовский классификатор (Naive Bayes), случайные леса (random forest) и метод опорных векторов (SVM).
Лучшая модель (наивный байесовский классификатор) показала результат метрики качества F1-мера: 0.94.  %! Проверить.
В работе \cite{Kotelnikov:1} за исключением предыдущих рассмотрены методы Rochio и
k-ближайших соседей (kNN, k-Nearest Neighbors). Авторами решалась задача многоклассовой классификации (на 5 классов), поэтому лучшая модель описывается с метрикой качества accuracy: 0.812. 
Работа \cite{Ghorbani:1} посвящена конкатенации двух нейросетевых концепций: сверточных нейронных сетей (одномерных)
и рекуррентных нейронных сетей: LSTM (долгая краткосрочная память, long short-term memory). Рассмотренная модель показала результат accuracy: 0.8902. %! Six feet under
Решению этой задачи нейросетевым моделированием посвящена и статья \cite{Xing:1}, в которой объектом для исследований стала 
архитектура рекуррентной нейронной сети: GRU (управляемый рекуррентный блок, gated recurrent unit). Метрика качества модели accuracy: 0.717.\\ %! The Human Paradox
Основной вклад конкретно этой работы заключается в конкатенации современных подходов к решению поставленной задачи, 
мета моделировании с целью повышения качества и уверенности моделей и создании виджета, не имеющего аналогов и олицетворяющего
законченность выпускной квалификационной работы. Целями написания служили структуризация имеющихся знаний в исследуемой области 
и последующий эмпирически обусловленный выбор модели для применения. Конечная мета-модель, обученная на объединенном датасете отзывов
с платформ IMDB и Amazon, общим объемом в 3.625 миллиона строк, оценивается метрикой качества accuracy в 0.965, что позволяет применять данную модель
для анализа общественных мнений путем скраппинга комментариев с сайтов YouTube, Twitter и/или, например,  Instagram. Кроме того, в работе рассмотрены приемы тематического моделирования LDA и BertTopic с целью дополнения ответа на вопрос: <<Каково общественное мнение?>> конкретизацией сферы его участия. Язык исследования: английский. 
Для адаптации результатов под другие языки необходима иного рода предобработка данных.